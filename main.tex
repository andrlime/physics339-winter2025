% packages
\documentclass[b5paper]{article}
\usepackage[utf8]{inputenc}
\usepackage[margin=1.1in]{geometry} % 0.9 for pset, 1.4 for livetex notes

% Math libraries
\usepackage{amsmath}
\usepackage{amssymb}
\usepackage{amsthm}
\usepackage{amsthm}

% Formatting
\usepackage{titling}
\usepackage{fancyhdr}
\usepackage{graphicx}
\usepackage{float}
\usepackage{titlesec}
\usepackage{setspace}
\usepackage[titles]{tocloft}

% Additional commands
\usepackage{cancel}
\usepackage{enumitem}
\usepackage{xfrac}
\usepackage{mathtools}

% Graphcs
\usepackage{tikz}
\usepackage{tcolorbox}
\usepackage{xcolor}
\usepackage{booktabs}
\usepackage{longtable}

% Content
\usepackage{physics}
\usepackage[version=4]{mhchem}
\usepackage{siunitx}
\usepackage{sectsty}
\usepackage{hyperref}

\hypersetup{
    colorlinks=true,
    linktoc=all,
    linkcolor=black,
}

\allsectionsfont{\sffamily}
\renewcommand{\cftsecfont}{\normalfont\bfseries\sffamily}
\renewcommand*{\contentsname}{Table of Contents}

\newcommand*\circled[1]{\tikz[baseline=(char.base)]{
            \node[shape=circle,draw,inner sep=2pt] (char) {#1};}}
\renewcommand\real{\mathbb{R}}
\newcommand\complex{\mathbb{C}}
\newcommand\integer{\mathbb{Z}}
\newcommand\rational{\mathbb{Q}}
\renewcommand\natural{\mathbb{N}}
\newcommand\probability{\mathbb{P}}
\newcommand\field{\mathbb{F}}
\newcommand\qsp{\qq{}}
\newcommand\curlyf{\mathcal{F}}
\newcommand\psp{(\Omega, \curlyf, \probability)}
\DeclarePairedDelimiter{\ceil}{\lceil}{\rceil}

\definecolor{DARKBLUE}{HTML}{005E7A}
\definecolor{BLACK}{HTML}{121212}
\definecolor{DARKRED}{HTML}{83223C}
\definecolor{ORANGE}{HTML}{F0B12C}
\definecolor{CYAN}{HTML}{2CF0CB}
\definecolor{RED}{HTML}{C41E3A}

\newcounter{lemmacounter}
\newcounter{definitioncounter}
\newtcolorbox{squarebox}{
  sharp corners,
  colback=yellow!5!white,
  colframe=DARKBLUE!75!BLACK,
  boxrule=0.5px
}

\newtcolorbox{squarebox2}{
  sharp corners,
  colback=yellow!5!white,
  colframe=ORANGE!75!BLACK,
  boxrule=0.5px
}

\newtcolorbox{roundbox1}{
  colback=yellow!5!white,
  colframe=BLACK!75!BLACK,
  boxrule=0.5px
}

\newtcolorbox{roundbox2}{
  colback=yellow!5!white,
  colframe=DARKRED!75!BLACK,
  boxrule=0.5px
}

\newtcolorbox{roundbox3}{
  colback=yellow!5!white,
  colframe=CYAN!75!BLACK,
  boxrule=0.5px
}

\newtcolorbox{roundbox4}{
  sharp corners,
  colback=RED!5!white,
  colframe=RED!75!BLACK,
  boxrule=0.5px
}

\newenvironment{problem}
  {\begin{roundbox1}}
  {\end{roundbox1}}
  
\newenvironment{solution}
  {\begin{proof}[Solution]}
  {\end{proof}}

\newenvironment{definition}
  {\begin{roundbox1}
  \stepcounter{lemmacounter}
  \textcolor{BLACK}{\textsf{\textbf{Definition \thesection.\arabic{lemmacounter}}}}}
  {\end{roundbox1}}

\newenvironment{example}
  {\begin{roundbox2}
  \stepcounter{lemmacounter}
  \textcolor{DARKRED}{\textsf{\textbf{Example \thesection.\arabic{lemmacounter}}}}}
  {\end{roundbox2}}

\newenvironment{lemma}
  {\begin{squarebox2}
  \stepcounter{lemmacounter}
  \textcolor{ORANGE}{\textsf{\textbf{Lemma \thesection.\arabic{lemmacounter}}}}}
  {\end{squarebox2}}

\newenvironment{aside}
  {\begin{roundbox4}
  \textcolor{RED}{\textsf{\textbf{Note}}}}
  {\end{roundbox4}}
  
\newenvironment{theorem}
  {\begin{squarebox}
  \stepcounter{lemmacounter}
  \textcolor{DARKBLUE}{\textsf{\textbf{Theorem \thesection.\arabic{lemmacounter}}}}}
  {\end{squarebox}}
  
\newenvironment{proposition}
  {\begin{roundbox3}
  \stepcounter{lemmacounter}
  \textcolor{DARKBLUE}{\textsf{\textbf{Proposition \thesection.\arabic{lemmacounter}}}}}
  {\end{roundbox3}}

\makeatletter
\DeclareRobustCommand{\@seccntformat}[1]{%
  \def\temp@@a{#1}%
  \def\temp@@b{subsubsection}%
  \ifx\temp@@a\temp@@b
  \textcolor{DARKRED}{\csname the#1\endcsname}%
  \quad
  \else
  \textcolor{DARKBLUE}{§\csname the#1\endcsname}%
  \quad
  \fi
} 
\makeatother

\numberwithin{equation}{section}
\numberwithin{lemmacounter}{section}
\numberwithin{definitioncounter}{section}

\begin{document}
    \newcommand{\courseshort}{\textbf{Physics 339-1}}
\newcommand{\coursetitle}{Quantum Mechanics 1}
\newcommand{\term}{Winter Quarter 2024}
\newcommand{\prof}{Gerald Gabrielse}
\newcommand{\textbook}{[textbook]}
\newcommand{\isbn}{[isbn]}

    \pagestyle{fancy}
    \fancyfoot[C]{}
    \fancyhead[L]{\textbf{Andrew Li} (\term)}
    \fancyfoot[L]{\courseshort \: \coursetitle}
    \fancyfoot[R]{\thepage}

    \title{\textbf{\textsf{\courseshort \, Lecture Notes}}}
    \author{Andrew Li}
    \date{\term}
    \clearpage\maketitle
    \noindent Original lecture notes for \textbf{\courseshort: \coursetitle}, from \term, taught by Professor \prof. This course follows \textbook, ISBN \isbn.
    \thispagestyle{empty}

    \tableofcontents
    \AddToHook{cmd/section/before}{\clearpage}
    
    \section{January 6, 2025}

    \include{contents/lecture2.tex}
    \include{contents/lecture3.tex}
    \include{contents/lecture4.tex}
    \include{contents/lecture5.tex}
    \include{contents/lecture6.tex}
    \include{contents/lecture7.tex}
    \include{contents/lecture8.tex}
    \include{contents/lecture9.tex}
    \include{contents/lecture10.tex}
    \include{contents/lecture11.tex}
    \include{contents/lecture12.tex}
    \include{contents/lecture13.tex}
    \include{contents/lecture14.tex}
    \include{contents/lecture15.tex}
    \include{contents/lecture16.tex}
    \include{contents/lecture17.tex}
    \include{contents/lecture18.tex}
    \include{contents/lecture19.tex}
    \include{contents/lecture20.tex}
    \include{contents/lecture21.tex}
    \include{contents/lecture22.tex}
    \include{contents/lecture23.tex}
    \include{contents/lecture24.tex}
    \include{contents/lecture25.tex}
    \include{contents/lecture26.tex}

\end{document}
